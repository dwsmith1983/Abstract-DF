\chapter{Introduction to Groups}

\section{Basic Axioms and Examples}
Let \(G\) be a group.
\begin{enumerate}
\item
  Determine which of the following binary operations are associative:
  \begin{enumerate}[label = (\alph*)]
  \item
    the operation \(\star\) on \(\mathbb{Z}\) defined by
    \(a\star b = a - b\)
    \par\smallskip
    To be associative, \(a\star (b\star c) = (a\star b)\star c\).
    Let \(a, b, c\in G\).
    Then
    \begin{align*}
      a\star (b\star c) & = a - (b - c)\\
                              & = a - b + c\\
                              & = (a\star b) + c\\
                              & \neq (a - b) - c\\
                              & = (a\star b)\star c
    \end{align*}
    Let \(a = 1\), \(b = 2\), and \(c = 3\).
    Then \(a\star (b\star c) = 2\) and \((a\star b)\star c = -4\).
    The binary operation is not associative.
  \item
    the operation \(\star\) on \(\mathbb{R}\) defined by
    \(a\star b = a + b + ab\)
    \par\smallskip
    Let \(a, b, c\in G\).
    Then
    \begin{align*}
      (a\star b)\star c & = a + b + ab + c + (a + b + ab)c\\
                        & = a + b + ab + c + ac + bc + abc\\
                        & = a + b + c + bc + ab + ac + abc\\
                        & = a + b + c + bc + a(b + c + bc)\\
                        & = a\star (b\star c)
    \end{align*}
    Therefore, the binary operation is associative.
  \item
    the operation \(\star\) on \(\mathbb{Q}\) defined by
    \(a\star b = \frac{a + b}{5}\)
    \par\smallskip
    Let \(a, b, c\in G\).
    Then
    \begin{align*}
      (a\star b)\star c & = \frac{\frac{a + b}{5} + c}{5}\\
                        & = \frac{a + b + 5c}{25}\\
                        & = \frac{b + 5c + a}{25}\\
                        & = \frac{\frac{b + 5c}{5} + \frac{a}{5}}{5}\\
                        & \neq a\star (b\star c)
    \end{align*}
    Let \(a = 1\), \(b = 2\), and \(c = 3\).
    Then \((a\star b)\star c = \frac{18}{25}\) and
    \(a\star (b\star c) = \frac{2}{5}\).
    The binary operation is not associative.
  \item
    the operation \(\star\) on \(\mathbb{Z}\times\mathbb{Z}\) defined by
    \((a, b)\star (c, d) = (ad + bc, bd)\).
    \par\smallskip
    Let \(a, b, c, d, e, f \in G\).
    Then
    \begin{align*}
      ((a, b)\star (c, d))\star (e, f) & = (ad + bc, bd)\star (e, f)\\
                                       & = ((ad + bc)f + bde, bdf)\\
                                       & = (adf + bcf + bde, bdf)\\
                                       & = (bcf + bde + adf, bdf)\\
                                       & = (adf + b(cf + de), bdf)\\
                                       & = (a, b)\star ((c, d)\star (e, f))
    \end{align*}
    The binary operation is associative.
  \item
    the operation \(\star\) on \(\mathbb{Q}\setminus\{0\}\) defined by
    \(a\star b = \frac{a}{b}\)
    \par\smallskip
    Let \(a, b, c\in G\).
    Then
    \begin{align*}
      (a\star b)\star c & = \frac{\frac{a}{b}}{c}\\
                        & = \frac{a}{bc}\\
                        & = \frac{\frac{a}{c}}{b}\\
                        & \neq a\star (b\star c)
    \end{align*}
    Let \(a = 1\), \(b = 2\), and \(c = 2\).
    Then \((a\star b)\star c = \frac{1}{6}\) and
    \(a\star (b\star c) = \frac{3}{2}\).
    The binary operation is not associative.
  \end{enumerate}
\item
  Decide which of the binary operations in the preceding exercise are
  commutative.
  \begin{enumerate}[label = (\alph*)]
  \item
    the operation \(\star\) on \(\mathbb{Z}\) defined by
    \(a\star b = a - b\)
    \par\smallskip
    To be commutative, \(a\star b = b\star a\).
    \begin{align*}
      a\star b & = a - b\\
               & = -(b - a)\\
               & = -(b\star a)\\
               & \neq b\star a
    \end{align*}
    Let \(a = 1\) and \(b = 2\).
    Then \(a\star b = -1\) and \(b\star a = 1\).
    The binary operation is not commutative.
  \item
    the operation \(\star\) on \(\mathbb{R}\) defined by
    \(a\star b = a + b + ab\)
    \begin{align*}
      a\star b & = a + b + ab\\
               & = b + a + ba\\
               & = b\star a
    \end{align*}
    The binary operation is commutative.
  \item
    the operation \(\star\) on \(\mathbb{Q}\) defined by
    \(a\star b = \frac{a + b}{5}\)
    \begin{align*}
      a\star b & = \frac{a + b}{5}\\
               & = \frac{b + a}{5}\\
               & = b\star a
    \end{align*}
    The binary operation is commutative.
  \item
    the operation \(\star\) on \(\mathbb{Z}\times\mathbb{Z}\) defined by
    \((a, b)\star (c, d) = (ad + bc, bd)\).
    \begin{align*}
      (a, b)\star (c, d) & = (ad + bc, bd)\\
                         & = (cb + da, db\\
                         & = (c, d)\star (a, b)
    \end{align*}
    The binary operation is commutative.
  \item
    the operation \(\star\) on \(\mathbb{Q}\setminus\{0\}\) defined by
    \(a\star b = \frac{a}{b}\)
    \begin{align*}
      a\star b & = \frac{a}{b}\\
      b\star a & = \frac{b}{a}
    \end{align*}
    Let \(a = 1\) and \(b = 2\).
    Then \(a\star b = \frac{1}{2}\) and \(b\star a = 2\).
    The binary operation is not commutative.
  \end{enumerate}
\item
  Prove that addition of residue classes in \(\mathbb{Z}/n\mathbb{Z}\) is
  associative (you may assume it is well defined).
  \par\smallskip
  Let \(\bar{a}, \bar{b}, \ldots, \overline{n - 1}\) be the residue classes of
  \(\mathbb{Z}/n\mathbb{Z}\).
  \begin{align*}
    (a\star b)\star c & = (\bar{a} + \bar{b}) + \bar{c}\\
                      & = \overline{a + b} + \bar{c}\\
                      & = \overline{a + b + c}\\
                      & = \bar{a} + (\overline{b + c})\\
                      & = \bar{a} + (\bar{b} + \bar{c})\\
                      & = a\star (b\star c)
  \end{align*}
  The binary operation of addition on \(\mathbb{Z}/n\mathbb{Z}\) is
  associative.
\item
  Prove that multiplication of residue classes in \(\mathbb{Z}/n\mathbb{Z}\) is
  associative (you may assume it is well defined).
  \par\smallskip
  Let \(\bar{a}, \bar{b}, \ldots, \overline{n - 1}\) be the residue classes of
  \(\mathbb{Z}/n\mathbb{Z}\).
  \begin{align*}
    (a\star b)\star c & = (\bar{a}\bar{b})\bar{c}\\
                      & = \overline{abc}\\
                      & = \bar{a}(\bar{b}\bar{c})\\
                      & = a\star (b\star c)
  \end{align*}
\item
  Prove for all \(n > 1\) that \(\mathbb{Z}/n\mathbb{Z}\) is not a group under
  multiplication of residue classes.
  \par\smallskip
  For \(n \geq 2\), the set of residue classes is
  \(S = \{x:x\text{ belongs to one of the reside classes, }\bar{0}, \bar{1}, \ldots, \overline{n - 1}\}\).
  Then for any \(a, b\in S\).
  Then \(\mathbb{Z}/n\mathbb{Z}\) is closed under multiplication since
  \(a\cdot b\equiv z\pmod{n}\) where \(z\) is an integer of lowest order in mod
  \(n\).
  The identity element is one since \(a\cdot 1 \equiv a\pmod{n}\).
  \(\mathbb{Z}/n\mathbb{Z}\) is associative by problem \(4\).
  In order for \(\mathbb{Z}/n\mathbb{Z}\) to be a group, we need to establish
  the existence of the inverse.
  We have that \(0\cdot a\equiv 0\pmod{n}\) for all \(a\in S\).
  That is, no element in the residue class of zero has an inverse.
  Therefore, \(\mathbb{Z}/n\mathbb{Z}\) is not a group.
\item
  Determine which of the following sets are groups under addition:
  \begin{enumerate}[label = (\alph*)]
  \item
    the set of all rational numbers (including \(0 = 0/1\)) in lowest terms
    whose denominators are odd
    \par\smallskip
    First, we need to determine if the set is closed under addition.
    \[
    \frac{a}{2b + 1} + \frac{c}{2d + 1} =
    \frac{a(2d + 1) + c(2b + 1)}{(2b + 1)(2d + 1)}
    \]
    The numerator is integer so the only worry is the denominator which needs
    to be odd.
    Now, \((2b + 1)(2d + 1) = 4bd + 2b + 2d + 1 = 2(2bd + b + d) + 1\) which is
    odd.
    Therefore, the set is closed under addition.
    Let \(e\) be the identity element.
    Then
    \[
    \frac{a}{2b + 1} + e = \frac{a}{2b + 1}\Rightarrow e = 0
    \]
    which establishes the existence of the identity element.
    Next, we need to show the existence of the inverse.
    Let \(x\) be the inverse element.
    Then
    \[
    \frac{a}{2b + 1} + x = e = 0\Rightarrow x = \frac{-a}{2b + 1}
    \]
    which establishes the existence of the inverse element.
    Let \(b\), \(d\), and \(f\) be odd.
    That is, \(b\), \(d\), and \(f\) are of the form \(2n + 1\).
    \begin{align*}
      (w\star y)\star z
      & = \Bigl(\frac{a}{b} + \frac{c}{d}\Bigr) + \frac{e}{f}\\
      &= \frac{ad + cb}{bd} + \frac{e}{f}\\
      &= \frac{(ad + cb)f + ebd}{dbf}\\
      &= \frac{adf + cbf + ebd}{bdf}\\
      &= \frac{adf + (cf + ed)b}{bdf}\\
      &= \frac{a}{b} + \Bigl(\frac{c}{d} + \frac{e}{f}\Bigr)\\
      & = w\star (y\star z)
    \end{align*}
    Therefore, the set is associative, and we can say is a group under
    addition.
  \item
    the set of rational numbers in lowest terms whose denominators are even
    together with zero
    \par\smallskip
    This set is not closed under addition since
    \(\frac{1}{6} + \frac{1}{6} = \frac{2}{3}\).
    Therefore, the set is not a group under addition.
  \item
    the set of rational numbers of absolute value \(< 1\)
    \par\smallskip
    This set is not closed under addition since
    \(\frac{1}{2} + \frac{3}{4} = \bigl\lvert\frac{5}{4}\bigr\rvert > 1\).
    Therefore, the set is not a group under addition.
  \item
    the set of rational numbers of absolute value \(\geq 1\) together with
    zero
    \par\smallskip
    This set is not closed under addition since
    \(-1 + \frac{3}{2} = \bigl\lvert\frac{1}{2}\bigr\rvert < 1\).
    Therefore, the set is not group under addition.
  \item
    the set of rational numbers with denominators equal to \(1\) or \(2\)
    \par\smallskip
    Let \(m = \frac{a}{1}\) and \(n = \frac{b}{2}\).
    Then \(m + n = \frac{a}{1} + \frac{b}{2} = \frac{2a + b}{2}\) which has
    a denominator of \(2\) if \(2a + b\) is odd.
    If \(2a + b\) is even, then we can write it as \(2r\) so
    \(m + n = \frac{r}{1}\) which has denominator one.
    Therefore, the set is close under addition.
    Let \(e\) be the identity and \(t\) belong to the set; that is, \(t\) has
    denominator of one or two.
    Then \(t + e = t\) so \(e = 0\).
    Let \(x\) be the inverse.
    Then \(t + x = e\) so \(x = -t\).
    Let \(a\), \(b\), and \(c\) be rationals that belong to the set.
    Since \(\mathbb{Q}\) is associative, this set is associative.
    Therefore, this set is a group under addition.
  \item
    the set of rational numbers with denominators equal to \(1\), \(2\), or
    \(3\)
    \par\smallskip
    This set is not closed under addition since
    \(\frac{1}{2} + \frac{1}{3} = \frac{5}{6}\).
    Therefore, this set is not a group under addition.
  \end{enumerate}
\item
  Let \(G = \{x\in\mathbb{R}\mid 0\leq x < 1\}\) and for \(x, y\in G\) let
  \(x\star y\) be the fractional part of \(x + y\) (that is,
  \(x\star y = x + y - [x + y]\) where \([a]\) is the greatest integer less
  than or equal to \(a\)).
  Prove that \(\star\) is a well defined binary operation on \(G\) and that
  \(G\) is an abelian group under \(\star\) (called the real numbers modulo
  one).
  \par\smallskip
  We have two cases to consider for \([x + y]\).
  Since \(x, y < 1\), we can have that
  \[
  [x + y] =
  \begin{cases}
    0, & x, y < 0.5\\
    1, & \text{if either \(x\), \(y\), or both are } > 0.5 
  \end{cases}
  \]
  For \([x + y] = 0\), we have that \(x, y < 0.5\) so \(0\leq x + y < 1\) and
  \(x\star y\in G\).
  For the second case, \(x + y < 2\) since \(x, y < 1\).
  Then \(x\star y = x + y - [x + y] < 2 - 1 = 1\) so \(x\star y\in G\).
  Hence, \(\star\) is well defined.
  Let \(e\) be the identity element.
  Then \(x\star e = x\).
  Let \(x\in G\).
  Then
  \begin{align*}
    x\star e & = x + e - [x + e]\\
             & = x + e\tag{for \([x + e] = 0\)}\\
    \intertext{Therefore, \(e = 0\).}
             & = x + e - 1\tag{for \([x + e] = 1\)}
  \end{align*}
  In the second case, we would get \(e = 1\) which clearly doesn't exist in
  \(G\) so \(e = 0\) is the identity element.
  Let \(v\) be the inverse element in \(G\).
  Then \(x\star v = e = 0\).
  \begin{align*}
    x\star v & = x + v - [x + v]\\
             & = x + v\tag{for \([x + v] = 0\)}\\
    \intertext{Therefore, \(v = -x\).}
             & = x + v - 1\tag{for \([x + v] = 1\)}
  \end{align*}
  In the second case, we get that \(v = 1 - x\in G\) since \(x\in G\).
  Recall that the identity element is unique.
  That is, if \(v = -x\), when \(x = 0\), \(v = 0\) and the inverse would be
  the identity element.
  Therefore, \(v = 1 - x\).
  Let \(x, y, z\in G\).
  Then
  \begin{align*}
    x\star (y\star z) & = x + (y\star z) - [x + (y\star z)]\\
                      & = x + y + z - [y + z] - [x + y + z - [y + z]]\\
                      & = x + y + z - [y + z] - [x + y + z] + [y + z]\eqnumtag
                        \label{ch1prob8}\\
                      & = x + y + z - [x + y + z]\\
                      & = x + y + z - [x + y] - [x + y + z] + [x + y]\\
                      & = x + y + z - [x + y] - [x + y + z - [x + y]]\\
                      &=  (x + y - [x + y]) + z - [(x + y - [x + y]) + z]\\
                      & = (x\star y) + z - [(x\star y) + z]\\
                      & = (x\star y)\star z
  \end{align*}
  \Cref{ch1prob8} occurs since if \(x\in\mathbb{R}\) and \(n\in\mathbb{Z}^+\),
  then \([x + n] = [x] + n\).
  Therefore, \(\star\) is associative.
  \begin{align*}
    x\star y & = x + y - [x + y]\\
             & = y + x - [y + x]\\
             & = y\star x
  \end{align*}
  Therefore, \(\star\) is commutative and \(G\) is an abelian group.
\item
  Let
  \(G = \{z\in\mathbb{C}\mid z^n = 1\text{ for some } n\in\mathbb{Z}^+\}\).
  \begin{enumerate}[label = (\alph*)]
  \item
    Prove that \(G\) is a group under multiplication (called the group of
    \textit{roots of unity} in \(\mathbb{C}\)).
    \par\smallskip
    Let \(z_1, z_2\in G\).
    Then there exist \(n, m\in\mathbb{Z}^+\) such that \(z_1^n = 1\) and
    \(z_2^m = 1\).
    Now, take \((z_1z_2)^{mn} = z_1^nz_2^m = 1\cdot 1 = 1\); therefore, \(G\)
    is closed under multiplication.
    Since \(1^1 = 1\), we have that \(1\in G\).
    With multiplication, \(1\cdot z^n = z^n\) for \(z^n\in G\).
    Thus, \(1 = e\) is the identity element in \(G\).
    Since \(\mathbb{C}\) is a field, multiplication is associative; hence,
    \(G\) is associative which we can easily show as well.
    \begin{align*}
      z_1^n(z_2z_3)^{pq} & = z_1^nz_2^pz_3^q\\
                         & = (z_1^nz_2^p)z_3^q\\
                         & = (z_1z_2)^{np}z_3^q
    \end{align*}
    Let be \(x\) the inverse element.
    Then 
    \begin{align*}
      z^nx & = e\\
      x & = z^{-n}\\
      z^nz^{-n} &= z^n\bigl(z^n\bigr)^{-1}\\
           & = 1\cdot 1^{-1}\\
           & = 1
    \end{align*}
    The inverse element \(x = z^{-n}\).
    Therefore, \(G\) is a group under multiplication; moreover, \(G\) is an
    abelian group since \(\mathbb{C}\) is a field and multiplication is
    commutative in \(\mathbb{C}\) so it is commutative in \(G\).
  \item
    Prove that \(G\) is not a group under addition.
    \par\smallskip
    Let \(z_1, z_2\in G\) and \(n, m\in\mathbb{Z}^+\).
    Then
    \[
    z_1^n + z_2^m = 1 + 1 = 2.
    \]
    Therefore, \(G\) is not close under addition so \(G\) cannot be a group.
  \end{enumerate}
\item
  Let \(G = \{a + b\sqrt{2}\in\mathbb{R}\mid a, b\in\mathbb{Q}\}\).
  \begin{enumerate}[label = (\alph*)]
  \item
    Prove that \(G\) is a group under addition.
    \par\smallskip
    Let \(a + b\sqrt{2}, c + d\sqrt{2}\in G\).
    Then \(a + b\sqrt{2} + c + d\sqrt{2} = a + c + (b + d)\sqrt{2}\in G\) since
    \(\mathbb{Q}\) is closed under addition so \(a + c, b + d\in\mathbb{Q}\).
    \(G\) is associative since \(\mathbb{R}\) is associate.
    \(0\in G\) since \(0 = 0 + 0\sqrt{2}\).
    Let \(x\in G\).
    Then \(x + 0 = x\) so \(0\) is the identity element \(e\in G\).
    For all \(a, b\in\mathbb{Q}\) and \(a + b\sqrt{2}\in G\), we have
    \(-a - b\sqrt{2}\in G\) and \(a + b\sqrt{2} - a - b\sqrt{2} = 0\);
    therefore, \(-a - b\sqrt{2}\) is the inverse element in \(G\).
    Hence, \(G\) is a group under addition.
  \item
    Prove that the nonzero elements of \(G\) are a group under multiplication.
    ("Rationalize the denominators" to find multiplicative inverses.)
    \par\smallskip
    Let \(a,b,c,d\in\mathbb{Q}\setminus\{0\}\).
    Then \(a + b\sqrt{2}, c + d\sqrt{2}\in G\).
    \[
    (a + b\sqrt{2})(c + d\sqrt{2}) = ac + 2bd + (bc + ad)\sqrt{2}
    \]
    and since \(ac + 2bd, bc + ad\in\mathbb{Q}\),
    \(ac + 2bd + (bc + ad)\sqrt{2}\in G\) so \(G\) is closed under
    multiplication.
    Since \(\mathbb{R}\) is associative, \(G\) is associative.
    \(1\in G\) since \(1 = 1 + 0\sqrt{2}\).
    For all \(a,b\in\mathbb{Q} - \{0\}\) and \(a + b\sqrt{2}\in G\),
    \((a + b\sqrt{2})(1) = a + b\sqrt{2}\).
    That is, \(1\) is the identity element \(e\in G\).
    \begin{align*}
      (a + b\sqrt{2})(x) & = e\\
      x & = \frac{1}{a + b\sqrt{2}}\\
                         & = \frac{a - b\sqrt{2}}{a^2 - 2b^2}\\
                         & = \frac{a}{a^2 - 2b^2} - \frac{b}{a^2 - 2b^2}
                           \sqrt{2}
    \end{align*}
    The inverse element is \(\frac{1}{a + b\sqrt{2}}\in G\) since
    \(\frac{a}{a^2 - 2b^2}, \frac{-b}{a^2 - 2b^2}\in\mathbb{Q}\) and
    \(a^2 - 2b^2 = 0\not\in\mathbb{Q}\) since
    \(\sqrt{2}\in\mathbb{R}\setminus\mathbb{Q}\).
  \end{enumerate}
\item
  Prove that a finite group is abelian if and only if its group table is a
  symmetric matrix.
  \par\smallskip
  We have to prove two statements.
  \begin{enumerate}[label = (\alph*)]
  \item
    If a finite group is abelian, then its group table is a symmetric matrix.
    \par\smallskip
    Let \(G\) be a finite group with \(\lvert G\rvert = n\) and
    \(g_i, g_j\in G\) for \(i\neq j\).
    The group table is the \(n\times n\) matrix whose \(i,j\) entry is the
    group element \(g_ig_j\).
    \[
    \begin{bmatrix}
      g_1g_1 & g_1g_2 & \cdots & g_1g_n\\
      g_2g_1 & g_2g_2 & \cdots & g_2g_n\\
      \vdots & & \ddots & \vdots\\
      g_ng_1 & g_ng_2 & \cdots & g_ng_n
    \end{bmatrix}
    \]
    Since \(G\) is abelian, \(g_ig_j = g_jg_i\).
    A symmetric matrix is \(\mathbf{A} = \mathbf{A}^{\intercal}\) or when
    \(a_{ij} = a_{ji}\).
    Since \(a_{ij} = g_ig_j = g_jg_i = a_{ji}\), the group table is symmetric.
    Thus, if a finite group is abelian, then its group table is symmetric.
  \item
    If a group table is a symmetric matrix, then its finite group is abelian.
    \par\smallskip
    Let \(\mathbf{A}\) be the symmetric \(n\times n\) group table matrix.
    Then \(a_{ij} = a_{ji}\).
    Let \(g_i,g_j\in G\) where \(g_ig_j = a_{ij}\).
    Since \(\mathbf{A}\) is symmetric, \(a_{ij} = g_ig_j = a_{ji} = g_jg_i\).
    Therefore, \(g_ig_j = g_jg_i\) so \(G\) is abelian.
    Additionally, since a symmetric matrix is finite and square,
    \(\lvert G\rvert = n\).
    If a group table is a symmetric matrix, then its finite group is abelian.
  \end{enumerate}
\item
  Find the orders of each element of the additive group
  \(\mathbb{Z}/12\mathbb{Z}\).
  \par\smallskip
  Let \(G\) be set congruence classes of \(\mathbb{Z}/12\mathbb{Z}\).
  Then \(G = \{\bar{0},\bar{1},\ldots,\overline{11}\}\).
  The order of \(\bar{0}\) is one since \(0\equiv 0\pmod{12}\).
  The order of \(\bar{1}\) is twelve since
  \(\underbrace{\bar{1} + \bar{1} + \cdots + \bar{1}}_{12\text{ times}}\equiv
  0\pmod{12}\).
  By similar means, we have that \(\lvert\bar{2}\rvert = 6\),
  \(\lvert\bar{3}\rvert = 4\), \(\lvert\bar{4}\rvert = 3\),
  \(\lvert\bar{5}\rvert = 12\), \(\lvert\bar{6}\rvert = 2\),
  \(\lvert\bar{7}\rvert = 12\), \(\lvert\bar{8}\rvert = 3\),
  \(\lvert\bar{9}\rvert = 4\), \(\lvert\overline{10}\rvert = 6\), and
  \(\lvert\overline{11}\rvert = 12\).
\item
  Find the orders of each elements of the multiplicative group
  \((\mathbb{Z}/12\mathbb{Z})^{\times}\colon\bar{1}, \overline{-1}, \bar{5},
  \bar{7}, \overline{-7}, \bar{13}\).
  \par\smallskip
  The order of \(\lvert\bar{1}\rvert = 1\) and the order of
  \(\lvert\overline{-1}\rvert = 2\).
  The order of the others are \(\lvert\bar{5}\rvert = 2\),
  \(\lvert\bar{7}\rvert = 2\), \(\lvert\overline{-7}\rvert = 2\),
  and \(\lvert\overline{13}\rvert = 1\).
\item
  Find the orders of each element of the additive group
  \((\mathbb{Z}/36\mathbb{Z})\colon\bar{1}, \bar{2}, \bar{6}, \bar{9},
  \overline{10}, \overline{12}, \overline{-1}, \overline{-10},
  \overline{-18}\).
  \begin{alignat*}{4}
    \lvert\bar{1}\rvert & = 36 & \qquad & \lvert\bar{2}\rvert &&{}= 18\\
    \lvert\bar{6}\rvert & = 6 & \qquad & \lvert\bar{9}\rvert &&{}= 4\\
    \lvert\overline{10}\rvert & = 18 & \qquad & \lvert\overline{12}\rvert
    &&{}= 3\\
    \lvert\overline{-1}\rvert & = 36 & \qquad & \lvert\overline{-10}\rvert
    &&{}= 18\\
    \lvert\overline{-18}\rvert & = 2
  \end{alignat*}
\item
  Find the orders of the following elements of the mutliplicative group
  \((\mathbb{Z}/36\mathbb{Z})^{\times}\colon\bar{1}, \overline{-1}, \bar{5},
  \overline{13}, \overline{-13}, \overline{17}\).
  \begin{alignat*}{4}
    \lvert\bar{1}\rvert & = 1 & \qquad & \lvert\overline{-1}\rvert &&{}= 2\\
    \lvert\bar{5}\rvert & = 6 & \qquad & \lvert\overline{13}\rvert &&{}= 6\\
    \lvert\overline{17}\rvert & = 2
  \end{alignat*}
\item
  Prove that
  \((a_1a_2\cdots a_n)^{-1} = a_n^{-1}a_{n - 1}^{-1}\cdots a_1^{-1}\) for all
  \(a_1,a_2,\ldots,a_n\in G\).
  \par\smallskip
  Since \(G\) is a group, \(a_ia_i^{-1} = a_i^{-1}a_i = e\) where \(e\) is the
  identity element.
  Let's multiple by \((a_1a_2\cdots a_n)\) on the left and right hand side.
  Then
  \begin{align*}
    (a_1a_2\cdots a_n)(a_1a_2\cdots a_n)^{-1}
    & = (a_1a_2\cdots a_n)a_n^{-1}a_{n - 1}^{-1}\cdots a_1^{-1}\\
    e & = a_1a_2\cdots a_na_n^{-1}a_{n - 1}^{-1}\cdots a_1^{-1}\\
    & = a_1a_2\cdots a_{n - 1}ea_{n - 1}^{-1}\cdots a_1^{-1}\\
    &= e
  \end{align*}
\item
  Let \(x\) be an element of \(G\).
  Prove that \(x^2 = 1\) if and only if \(\lvert x\rvert\) is either one or
  two.
  \par\smallskip
  First, let's consider if \(x^2 = 1\), then \(\lvert x\rvert\) is either one
  or two.
  Since \(x^2\) is the multiplicative identity element, the maximum order of
  \(x\) is two.
  However, if the order of \(x\) is one, then \(1^2 = 1\).
  That is, the order of \(x\) can be either one or two.
  Now, suppose that if \(\lvert x\rvert\) is either one or two, then
  \(x^2 = 1\).
  If the of order of \(x\) is two, then \(x^2 = 1\).
  If the of order of \(x\) is one, then \(x^1 = 1\) so
  \(x^2 = x^1x^1 = (1)(1) = 1\).
  Thus, \(x^2 = 1\).
\item
  Let \(x\) be an element of \(G\).
  Prove that if \(\lvert x\rvert = n\) for some positive integer \(n\) then
  \(x^{-1} = x^{n - 1}\).
  \par\smallskip
  Since \(\lvert x\rvert = n\), \(x^n = e\) where \(e\) is the identity
  element.
  Let's multiple by \(x^{-1}\) on the right and left so we have
  \[
  x^{n}x^{-1} = ex^{-1}\Rightarrow x^{n - 1} = x^{-1}.
  \]
\item
  Let \(x,y\in G\).
  Prove that \(xy = yx\) if and only if \(y^{-1}xy = x\) if and only if
  \(x^{-1}y^{-1}xy = 1\).
  \par\smallskip
  Since \(x,y\in G\), we have that \(x^{-1},y^{-1}\in G\) and
  \(yy^{-1} = y^{-1}y = e\) where \(e\) is the identity element so
  \begin{align*}
    xy & = yx\\
    y^{-1}xy & = y^{-1}yx\tag{mulitple by \(y^{-1}\) on the left}\\
    y^{-1}xy & = ex\\
    y^{-1}xy & = x\\
    x^{-1}y^{-1}xy & = 1\tag{mulitple by \(x^{-1}\) on the left}
  \end{align*}
  To prove the other direction, we simply start from \(x^{-1}y^{-1}xy = 1\) and
  work back up since \(x,y\in G\).
\item
  Let \(x\in G\) and let \(a,b\in\mathbb{Z}^+\).
  \begin{enumerate}[label = (\alph*), ref = \theenumi (\alph*)]
  \item
    \label{19a}
    Prove that \(x^{a + b} = x^ax^b\) and \((x^a)^b = x^{ab}\).
    \par\smallskip
    
  \item
    Prove that \((x^a)^{-1} = x^{-a}\).
  \item
    Establish \cref{19a} for arbitary integers \(a\) and \(b\) (positive,
    negative or zero).
  \end{enumerate}
\item
  For \(x\) an element in \(G\) show that \(x\) and \(x^{-1}\) have the same
  order.
  \par\smallskip
  Let \(\lvert x\rvert = n\).
  Then \(x^n = e\).
  Since \(x^n\in G\), \(x^{-n}\in G\).
  Then
  \[
  x^{-n}x^n = x^{-n}e\Rightarrow e = x^{-n} = (x^{-1})^n
  \]
  Therefore, \(\lvert x^{-1}\rvert = n\).
\item
  Let \(G\) be a finite group and let \(x\) be an element of \(G\) of order
  \(n\).
  Prove that if \(n\) is odd, then \(x = (x^2)^k\) for some integer
  \(k\geq 1\).
  \par\smallskip
  Since \(n\) is odd, we can wrie \(n = 2k - 1\) where \(k\in\mathbb{Z}\).
  Now, \(x^n = x^{2k - 1} = x^{2k}x^{-1} = e\) so \(x^{2k} = x\).
\item
  If \(x\) and \(g\) are elements of the group \(G\), prove that
  \(\lvert x\rvert = \lvert g^{-1}xg\rvert\).
  Deduce that \(\lvert ab\rvert = \lvert ba\rvert\) for all \(a,b\in G\).
  \par\smallskip
  Let \(\lvert x\rvert = n\).
  Then
  \begin{align*}
    x^n & = (g^{-1}xg)^n\\
        & = \underbrace{(g^{-1}xg)\cdots (g^{-1}xg)}_{n\text{ times}}\\
        & = g^{-1}x^ng\\
        & = g^{-1}eg\\
        & = g^{-1}g\\
        & = e
  \end{align*}
  Thus, \(\lvert g^{-1}xg\rvert = n = \lvert x\rvert\).
  Now, suppose \(\lvert x\rvert = \infty\) and \(\lvert g^{-1}xg\rvert = n\).
  Then
  \[
  g^{-1}x^ng = e\Rightarrow gg^{-1}x^ngg^{-1} = geg^{-1}\Rightarrow
  x^n = e
  \]
  which is a contradiction.
  That is, if \(\lvert x\rvert = \infty\), then so does
  \(\lvert g^{-1}xg\rvert\).
  From above, we have that \(\lvert ab\rvert = \lvert g^{-1}(ab)g\rvert\).
  \begin{align*}
    \lvert ab\rvert & = \lvert g^{-1}(ab)g\rvert\tag{Let \(g = b^{-1}a\)}\\
                    & = \lvert ba^{-1}(ab)b^{-1}a\rvert\\
                    & = \lvert ba\rvert
  \end{align*}
\item
  Suppose \(x\in G\) and \(\lvert x\rvert = n < \infty\).
  If \(n = st\) for some positive integers \(s\) and \(t\), prove that
  \(\lvert x^s\rvert = t\).
  \par\smallskip
  Since the order of \(x\) is \(n\), we have
  \begin{align*}
    x^n & = x^{st}\\
        & = (x^s)^t\\
    \lvert x^s\rvert & = t\tag{since \(x^n = (x^s)^t = e\)}
  \end{align*}
\item
  If \(a\) and \(b\) are \textit{commuting} elements of \(G\), prove that
  \((ab)^n = a^nb^n\) for all \(n\in\mathbb{Z}\).
  (Do this by induction for positive \(n\) first.)
  \par\smallskip
  Since \(a\) and \(b\) commute, \(ab = ba\).
  Let \(n = 1\).
  Then \((ab)^1 = ab\).
  Suppose this is true for \(k\leq n\).
  Then \((ab)^k = a^kb^k\).
  \begin{align*}
    (ab)^k(ab) & = a^kb^kab\\
               & = a^k\underbrace{b\cdots b}_{k\text{ times}}ab\\
               & = a^k\underbrace{b\cdots b}_{k - 1\text{ times}}abb\\
               & = \vdots\\
               & = a^kbab^{k - 1}b\\
               & = a^kabb^k\\
               & = a^{k + 1}b^{k + 1}
  \end{align*}
  By the prinicple of mathematical induction, \((ab)^n = a^nb^n\) for all
  \(n\in\mathbb{Z}^+\).
  For any \(n < 0\), we have
  \begin{align*}
    (ab)^n & = ((ab)^{-n})^{-1}\\
           & = (a^{-n}b^{-n})^{-1}\\
           & = a^nb^n
  \end{align*}
\item
  Prove that if \(x^2 = 1\) for all \(x\in G\) then \(G\) is abelian.
  \par\smallskip
  Since \(x^2 = 1\), we have
  \[
  x^2 = xx = e\Rightarrow x = x^{-1}
  \]
  Therefore, for all \(x\in G\), \(x = x^{-1}\).
  Let \(x,y\in G\).
  Then
  \begin{align*}
    xy & = (xy)^{-1}\\
       & = y^{-1}x^{-1}\\
       & = yx\tag{since \(x = x^{-1}\)}
  \end{align*}
  Thus, \(G\) is abelian.
\item
  Assume \(H\) is a nonempty subset of \((G, \star)\) which is closed under the
  binary operation on \(G\) and is closed under inverses, that is, for all
  \(h,k\in H\), \(hk, h^{-1}\in H\).
  Prove that \(H\) is a group under the operation \(\star\) restricted to
  \(H\) (such a subset \(H\) is a called a \textit{subgroup} of \(G\)).
  \par\smallskip
  Since \(h,k\in H\), \(h\star k\in H\).
  Therefore, \(H\) is closed under the operation of star.
  Let \(e\) be the identity element.
  Then \(e = hh^{-1} = h^{-1}h\in H\) where \(h^{-1}\) is the inverse.
  Let \(h,k,m\in H\).
  \begin{align*}
    (h\star k)\star m & = (hk)\star m\\
                      & = (hk)m\\
                      & = hkm\\
                      & = h(km)\\
                      & = h\star (km)\\
                      & = h\star (k\star m)
  \end{align*}
  Thus, \(H\) is associative under \(\star\), and a subgroup since for all
  \(h\in H\), \(h^{-1}\) exist, \(H\) is closed under star, associative, and
  \(e\in H\).
\item
  Prove that if \(x\in G\) then \(\{x^n\mid n\in\mathbb{Z}\}\) is a subgroup of
  \(G\) (called the \textit{cyclic subgroup} of \(G\) generated by \(x\)).
  \par\smallskip
  Let \(H\) be the cyclic subgroup.
  Since \(0\in\mathbb{Z}\), \(x^0 = 1\in H\) so \(H\) is not empty.
  Let \(x^n,x^m\in H\).
  Then \(x^nx^m = x^{n + m}\in H\) so \(H\) is closed.
  Since \(G\) is a group and \(x\in G\), \(x^{-1}\in G\).
  Then since \(x^n\in H\), we have \((x^n)^{-1} = x^{-n}\in H\) where
  \(x^{-n}\) is the inverse element.
  Now \(x^nx^{-n} = x^{-n}x^n = e\in H\) where \(e\) is the identity element.
  \begin{align*}
    (x^nx^m)x^t & = (x^{n + m})x^t\\
                & = x^{n + m}x^t\\
                & = x^{n + m + t}\\
                & = x^n(x^{m + t})\\
                & = x^n(x^mx^t)
  \end{align*}
  Therefore, \(H\) is non empty, closed, posses both an identity and inverse
  elements, and is associative so \(H\) is a subgroup of \(G\).
\item
  Let \((A, \star)\) and \((B, \diamond)\) be groups and let \(A\times B\) be
  their direct product (as defined in example \(6\)).
  Verify all the group axioms for \(A\times B\).
  \begin{enumerate}[label = (\alph*)]
  \item
    prove that the associative law holds: for all \((a_i, b_i)\in A\times B\),
    \(i = 1\), \(2\), \(3\)
    \((a_1, b_1)\bigl[(a_2, b_2)(a_3, b_3)\bigr] =
    \bigl[(a_1, b_1)(a_2, b_2)\bigr](a_3, b_3)\),
    \begin{align*}
      (a_1, b_1)\bigl[(a_2, b_2)(a_3, b_3)\bigr]
      & = (a_1, b_1)(a_2a_3, b_2b_3)\\
      & = (a_1a_2a_3, b_1b_2b_3)\\
      & = ((a_1a_2)a_3, (b_1b_2)b_3)\\
      & = \bigl[((a_1a_2), (b_1b_2))\bigr](a_3, b_3)\\
      & = \bigl[(a_1, b_1)(a_2, b_2)\bigr](a_3, b_3)
    \end{align*}
  \item
    prove that \((1, 1)\) is the identity of \(A\times B\), and
    \par\smallskip
    Let \(a,b\in A\times B\).
    Then \((a, b)(1, 1) = (a\cdot 1, b\cdot 1) = (a, b)\) and
    \((1, 1)(a, b) = (1\cdot a, 1\cdot b) = (a , b)\).
    Thus, \((1, 1)\) is the identity element in \(A\times B\).
  \item
    prove that the inverse of \((a, b)\) is \((a^{-1}, b^{-1})\).
    \par\smallskip
    Let \(a,b\in A\times B\).
    Then \((a, b)(a^{-1}, b^{-1}) = (aa^{-1}, bb^{-1}) = (1, 1)\) and
    \((a^{-1}, b^{-1})(a, b) = (a^{-1}a, b^{-1}b) = (1 , 1)\).
    Thus, \((a^{-1}, b^{-1})\) is the identity element in \(A\times B\).
  \end{enumerate}
\item
  Prove that \(A\times B\) is an abelian group if and only if both \(A\) and
  \(B\) are abelian.
  \par\smallskip
  Suppose that \(A\times B\) is abelian and let \(a,b\in A\) and
  \(\alpha,\beta\in B\).
  Since \(A\times B\) is abelian, for all
  \((a, \alpha), (b, \beta)\in A\times B\), we have
  \[
  (a, \alpha)(b, \beta) = (ab, \alpha\beta) = (b, \beta)(a, \alpha) =
  (ba, \beta\alpha)
  \]
  so \((ab, \alpha\beta) = (ba, \beta\alpha)\).
  Since \(ab = ba\) and \(a,b\in A\), \(A\) is abelian since \(a\) and \(b\)
  commute.
  Similarly, \(B\) is abelian since \(\alpha\beta = \beta\alpha\).
  Now suppose that \(A\) and \(B\) are abelian where \(a,b\in A\) and
  \(\alpha,\beta\in B\).
  Then \(ab = ba\) and \(\alpha\beta = \beta\alpha\).
  \begin{align*}
    (a, \alpha)(b, \beta) & = (ab, \alpha\beta)\\
                          & = (ba, \beta\alpha)
                            \tag{since \(A\) and \(B\) are abelian}\\
                          & = (b, \beta)(a, \alpha)
  \end{align*}
  Thus, \(A\times B\) is abelian.
\item
  Prove that the elements \((a, 1)\) and \((1, b)\) of \(A\times B\) commute
  and deduce that the order of \((a, b)\) is the least common multiple of
  \(\lvert a\rvert\) and \(\lvert b\rvert\).
  \begin{align*}
    (a, 1)(1, b) & = (a\cdot 1, 1\cdot b)\\
                 & = (a, b)\\
                 & = (1\cdot a, b\cdot 1)\\
                 & = (1, b)(a, 1)
  \end{align*}
  Hence, \((a, 1)\) and \((1, b)\) commute.
  Let the order of \(\lvert (a, 1)\rvert = n\), \(\lvert (1, b)\rvert = m\),
  and \(\lvert (a, 1)(1, b)\rvert = r\).
  Let \(d = [a, b]\) where \([a, b]\) is the LCM.
  Since \(d = [a, b]\), \(d\mid a\) and \(d\mid b\) or \(at = d\) and
  \(bs = d\).
  Then
  \[
  \bigl[(a, 1)(1, b)\bigr]^d = (a, 1)^d(1, b)^d = (1, 1)
  \]
  Suppose \(d\neq r\).
  Then \(r < d\) such that \(d\mid r\Rightarrow rh = d\) and
  \((a, 1)^r(1, b)^r = (1, 1)\).
  Since \((a, 1)^r = (1, 1)\) and \((1, b)^r = (1, 1)\), \(r\mid a\) and
  \(r\mid b\).
  Thus, \(r\) is a multiple of \(a\) and \(b\), but \(r < d\) which contradicts
  the fact that \(d\) is LCM.
  Therefore, \(r = d\) and the LCM is the order of \((a, b)\).
\item
  Prove that any finite group \(G\) of even order contains an element of order
  \(2\).
  (Let \(t(G)\) be the set \(\{g\in G\mid g\neq g^{-1}\}\).
  Show that \(t(G)\) has an even number of elements and every nonidentity
  element of \(G - t(G)\) has order \(2\).)
\item
  If \(x\) is an element of finite order \(n\) in \(G\), prove that the
  elements \(1,x,x^2,\ldots,x^{n - 1}\) are all distinct.
  Deduce \(\lvert x\rvert\leq\lvert G\rvert\).
  \par\smallskip
  Let \(x\in G\) and \(0\leq m < r \leq n - 1\).
  Suppose \(x^m = x^r\).
  Then \(x^mx^{-r} = e\) or \(x^{m - r} = e\).
  Since \(m,r\in\{0,1,\ldots,n - 1\}\) and \(m < r\), \(m - r < n\), thus
  the order of \(\lvert x\rvert\neq n\) which is a contradicition.
  No two elements are of the same order so they are distinct and \(G\) has at
  least \(n\) elements.
  Therefore, \(\lvert x\rvert\leq\lvert G\rvert\).
\item
  Let \(x\) be an element of finite order \(n\) in \(G\).
  \begin{enumerate}[label = (\alph*)]
  \item
    Prove that if \(n\) is odd then \(x^i\neq x^{-i}\) for all
    \(i = 1,2,\ldots, n - 1\).
  \item
    Prove that if \(n = 2k\) and \(1\leq i < n\) then \(x^i = x^{-i}\) if and
    only if \(i = k\).
  \end{enumerate}
\item
  If \(x\) is an element of infinite order in \(G\), prove that the elements
  \(x^n\), \(n\in\mathbb{Z}\) are all distinct.
  \par\smallskip
\item
  If \(x\) is an element of finite order \(n\) in \(G\), use the Division
  Algorithm to show that any integral power of \(x\) equals one of the elements
  in the set \(\{1,x,x^2,\ldots x^{n - 1}\}\) (so these are all the distinct
  elements of the cyclic subgroup of \(G\) generated by \(x\)).
\item
  Assume \(G = \{1, a, b, c\}\) is a group of order \(4\) with identity \(1\).
  Assume also that \(G\) has no elements of order \(4\) (so by exercise \(32\),
  every element has order \(\leq 3\)).
  Use the cancellation laws to show that there is a unique group table for
  \(G\).
  Deduce that \(G\) is abelian.
\end{enumerate}
%%% Local Variables:
%%% mode: latex
%%% TeX-master: t
%%% End:
