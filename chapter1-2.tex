\section{Dihedral Groups}
In these exercises, \(D_{2n}\) has the usual presentation
\(D_{2n} = \langle r,s\mid r^n = s^2 = 1, rs = sr^{-1}\rangle\).
\begin{enumerate}
\item
  Compute the order of each of the elements in the following groups:
  \begin{enumerate}[label = (\alph*)]
  \item
    \(D_6\)
    \par\smallskip
    The elements of \(D_6\) are \(\{1,r,r^2,s,sr,sr^2\}\).
    The first four elements orders are obviously \(\lvert 1\rvert = 1\),
    \(\lvert r\rvert = 2\), \(\lvert r^2\rvert = 3\), and
    \(\lvert s\rvert = 2\).
    Now, we have that
    \[
    (sr)(sr) = s(rs)r = s(sr^{-1})r = s^2 = 1
    \]
    so \(\lvert sr\rvert = 2\) and
    \[
    (sr^2)(sr^2) = s(sr^{-2})r^2 = 1
    \]
    so \(\lvert sr^2\rvert = 2\) as well.
  \item
    \(D_8\)
    \par\smallskip
    The elements of \(D_8\) are \(\{1,r,r^2,r^3,s,sr,sr^2,sr^3\}\).
    \begin{table}[H]
      \begin{alignat*}{3}
        \lvert 1\rvert & = 1 & \qquad \lvert s\rvert &&{}= 2\\
        \lvert r\rvert & = 4 & \qquad \lvert sr\rvert &&{}= 2\\
        \lvert r^2\rvert & = 2 & \qquad \lvert sr^2\rvert &&{}= 2\\
        \lvert r^2\rvert & = 4 & \qquad \lvert sr^3\rvert &&{}= 2
      \end{alignat*}
      \vspace{-0.75cm}
      \caption{Orders of the elements in \(D_8\).}
    \end{table}
  \item
    \(D_{10}\)
    \par\smallskip
    The elements of \(D_{10}\) are \(\{1,r,r^2,r^3,r^4,s,sr,sr^2,sr^3,sr^4\}\).
    The order of the elements in \(D_{10}\) are \(\lvert 1\rvert = 1\),
    \(\lvert r^i\rvert = 5\) for \(i = 1,\ldots,4\) and
    \(\lvert sr^j\rvert = 2\) for \(j = 0,\ldots,5\).
  \end{enumerate}
\item
  Use the generators and relations above to show that if \(x\) is any element
  of \(D_{2n}\) which is not a power of \(r\), then \(rx = xr^{-1}\).
  \par\smallskip
  Since \(x\in D_{2n}\), \(x = s^ir^j\) where \(i = 0,1\) and
  \(j = 0,\ldots,n\).
  If \(i = 0\), then \(x\) is a power of \(r\) so \(i = 1\).
  \[
  rx = rsr^j = sr^{-1}r^j = sr^jr^{-1} = xr^{-1}
  \]
\item
  \label{ex3}
  Use the generators and relations above to show that every element of
  \(D_{2n}\) which is not a power of \(r\) has order \(2\).
  Deduce that \(D_{2n}\) is generated by the two elements \(s\) and \(sr\),
  both of which have order \(2\).
  \par\smallskip
  Let \(x\) be an element of \(D_{2n}\) which is not a power of \(r\).
  Then \(x = sr^m\) where \(0\leq m < n\).
  \[
  (sr^m)(sr^m) = s(r^ms)r^m = s(sr^{-m})r^m = 1
  \]
  Thus, every element of \(D_{2n}\) which is not a power of \(r\) has order
  \(2\).
  Suppose that \(\langle s,sr\rangle\) are generators for \(D_{2n}\) and we
  know that \(\langle r,s\rangle\) are generators.
  It is obvious that \(s\in\langle s,sr\rangle\).
  Now \(s(sr) = r\in\langle s,sr\rangle\) so \(\langle s,sr\rangle\) generate
  \(D_{2n}\).
\item
  \label{ex4}
  If \(n = 2k\) is even and \(n\geq 4\), show that \(z = r^k\) is an element of
  order \(2\) which commutes with all elements of \(D_{2n}\).
  Show also that \(z\) is the only nonindentity element of \(D_{2n}\) which
  commutes with all elements of \(D_{2n}\).
  \par\smallskip
  Since \(z = r^k\in D_{2n}\), we have that
  \(r^kr^k = r^{2k} = (r^k)^2 = 1\) so \(\lvert z\rvert = 2\).
  Let \(x = s^ir^j\) where \(i = 0,1\) and \(0\leq j < n\).
  When \(i = 0\), \(x = r^j\).
  \[
  xr^k = r^jr^k = r^{j + k} = r^{k + j} = r^kr^j = r^kx
  \]
  so \(r^k\) commutes with \(x\) when \(i = 0\).
  Now let \(i = 1\).
  Then \(x = sr^j\).
  Recall that \(r^kr^k = 1\) so \(r^k = r^{-k}\).
  \[
  xr^k = sr^jk^k = sr^{k + j} = sr^kr^j = r^{-k}sr^j = r^kx
  \]
  Thus, \(r^k\) commutes with \(x\) for \(i = 0,1\).
  Suppose \(x = s^ir^m\) commutes with all elements of \(D_{2n}\) where
  \(i = 0,1\).
  When \(i = 1\), \(x = sr^m\).
  \[
  sr^mr = sr^{m + 1}\quad\text{and}\quad rsr^m = sr^{-1}r^m = sr^{m - 1}
  \]
  so \(m + 1 \equiv m - 1\pmod{n}\) so both \(m + 1\) and \(m - 1\) need to be
  a mutliple of \(n\).
  \[
  2\equiv 0\pmod{n}
  \]
  but \(n\geq 4\) so \(2\not\equiv 0\pmod{n}\).
  When \(i = 1\), \(sr^m\) doesn't commute with \(r\).
  Now let \(i = 0\).
  Then \(x = r^m\).
  \[
  sr^m = r^ms = r^{-m}s
  \]
  so \(2m\equiv 0\pmod{n}\); therefore, \(m = 0\) or \(2m = nt\).
  If \(m = 0\), then \(r^m = 1\) which is the identity element.
  Now when \(2m = nt\), \(0\leq m < n\) so \(t = 1\) and \(2m = n\).
  Since \(2m = n\), \(m = k\).
  So \(sr^k = r^{-k}s = r^ks\).
  Thus, the only elements that commute in \(D_{2n}\) are the identity and
  \(r^k\).
\item
  If \(n\) is odd and \(n\geq 3\), show that the identity is the only element
  of \(D_{2n}\) which commutes with all elements of \(D_{2n}\).
  \par\smallskip
  From \cref{ex4}, we know that for \(s^ir^m\) must statisfy
  \(2m\equiv 0\pmod{n}\) for \(n\geq 3\).
  Therefore, \(m = 0\) or \(2m = nt\) so again \(t = 1\) since \(0\leq m < n\).
  \[
  2m = n = 2p + 1
  \]
  which is a contradiction since \(2m\) is even and \(2p + 1\) is odd for
  \(p\in\mathbb{Z}\).
  Thus, the only element that commutes is the identity element.
\item
  Let \(x\) and \(y\) be elements of order two in and group \(G\).
  Prove that if \(t = xy\) then \(tx = xt^{-1}\) (so that if
  \(n = \lvert xy\rvert < \infty\) then \(x,t\) satisfy the same relations on
  \(G\) as \(s,r\) do in \(D_{2n}\)).
  \par\smallskip
  Since \(\lvert x\rvert = \lvert y\rvert = 2\), we have that \(x = x^{-1}\)
  and \(y = y^{-1}\).
  \[
  tx = xyx = xy^{-1}x^{-1} = xt^{-1}
  \]
\item
  Show that \(\langle a,b\mid a^2 = b^2 = (ab)^n = 1\rangle\) gives a
  presentation for \(D_{2n}\) in terms of the two generators \(a = s\) and
  \(b = sr\) of order two computed in \cref{ex3}.
  (Show that the relations for \(r\) and \(s\) follow from the relations for
  \(a\) and \(b\) and, conversely, the relations for \(a\) and \(b\) follow
  from those for \(r\) and \(s\).)
  \par\smallskip
  Since \(a = s\) and \(b = sr\), we have that \(a^2 = s^2 = 1\) and
  \(b^2 = (sr)(sr) = 1\), respectively.
  Now \(ab = ssr = r\) so \((ab)^n = r^n = 1\).
  \[
  r(sr) = abb = ab^2 = a = s\Rightarrow rsr = s\Rightarrow rs = sr^{-1}
  \]
  Now \(b^2 = (sr)(sr) = s(rs)s s(sr^{-1})r = 1\).
\item
  Find the order of the cyclic subgroup \(D_{2n}\) generated by \(r\).
  \par\smallskip
  The order of \(r\) in \(D_{2n}\) is \(\lvert r\rvert = n\) so
  \(\langle r\rangle = \{1,r,r^2,\ldots,r^{n - 1}\}\).
  Then \(\lvert\langle r\rangle\rvert = n\).
\end{enumerate}
In each of the exercise \(9\) to \(13\), you can find the order of the group
of rigid motions in \(\mathbb{R}^3\) (also called the group of rotations) of
the given Platonic solid by following the proof for the order of \(D_{2n}\):
find the number of positions to which an adjacent pair of vertices can be sent.
Alternatively, you can find the number of places to which a given  face may be
sent and, once a face is fixed, the number of positions which a vertex on that
face may be sent.
\begin{enumerate}[resume]
\item
  Let \(G\) be the group of rigid motions in \(\mathbb{R}^3\) of a tetrahedron.
  Show that \(\lvert G\rvert = 12\).
  \begin{figure}[H]
    \centering
    \includestandalone[mode = image, width = 2.5in]{Tikz/tetrahedron}
    \caption{Tetrahedron}
    \label{tetrahedron}
  \end{figure}
  Let's consider the point \(A\).
  Then \(A\) has four possible places it can be sent which includes the
  original location of \(A\).
  Once \(A\) is determined, \(B\) has three possibilities.
  With \(A\) and \(B\) determined, \(C\) and \(D\) are determined.
  Therefore, the order of \(G\) is \(\lvert G\rvert = 4\cdot 3 = 12\).
\item
  Let \(G\) be the group of rigid motions in \(\mathbb{R}^3\) of a cube.
  Show that \(\lvert G\rvert = 24\).
  \par\smallskip
  With a cube, let vertices \(1,2,3,4\) be in plane.
  Then vertice \(1\) has eight possible moves.
  Since \(2\) is in plane with one, it has only three possible moves.
  Once vertice \(1\) and \(2\) are determine, the other vertices are
  determined.
  The order \(G\) is \(\lvert G\rvert = 8\cdot 3 = 24\).
\item
  Let \(G\) be the group of rigid motions in \(\mathbb{R}^3\) of a octahedron.
  Show that \(\lvert G\rvert = 24\).
  \par\smallskip
  An octahedron has \(6\) points and \(8\) faces.
  Let vertices \(1,2,3\) be in plane.
  Vertice \(1\) can be sent to six locations.
  Once \(1\) is determine, \(2\) has four possible locations.
  With \(1\) and \(2\) determined, the other vertices are determined.
  Thus, the order of \(G\) is \(\lvert G\rvert = 6\dot 4 = 24\).
\item
  Let \(G\) be the group of rigid motions in \(\mathbb{R}^3\) of a
  dodecahedron.
  Show that \(\lvert G\rvert = 60\).
  \par\smallskip
  A dodecahedron has \(12\) faces and \(20\) vertices.
  Let vertices \(1,2,3,4,5\) be in plane.
  Vertice \(1\) can be sent to \(20\) locations.
  Vertice \(2\) can be set to three locations.
  After \(2\) is determined along with one, the other vertices are determined.
  Then the order of \(G\) is \(\lvert G\rvert = 20\cdot 3 = 60\).
\item
  Let \(G\) be the group of rigid motions in \(\mathbb{R}^3\) of a icosahedron.
  Show that \(\lvert G\rvert = 60\).
\item
  Find a set of generators for \(\mathbb{Z}\).
  \par\smallskip
  Every integer can be generated by repeated addition of \(\pm 1\).
  Therefore, \(\mathbb{Z} = \langle 1\rangle\).
\item
  Find a set of generators and relations for \(\mathbb{Z}/n\mathbb{Z}\).
  \par\smallskip
  The elements of \(\mathbb{Z}/n\mathbb{Z}\) are the classes
  \(\bar{0},\bar{1},\ldots,\overline{n - 1}\) which are mutliplies of
  \(\bar{1}\).
  The generator and relation is
  \(\mathbb{Z}/n\mathbb{Z} = \langle z\mid z^n = 1\rangle\).
\item
  Show that the group
  \(\langle x_1,y_1\mid x_1^2 = y_1^2 = (x_1y_1)^2 = 1\rangle\) is the diheral
  group \(D_4\) (where \(x_1\) may be replaced by the letter \(r\) and \(y_1\)
  by the letter \(s\)).
  [Show that the last relation is the same as: \(x_1y_1 = y_1x_1^{-1}\).]
  \par\smallskip
  In \(D_4\), the order of \(r\) is two.
  We have that \(x_1^2 = r^2 = 1\) and \(y_1^2 = s^2 = 1\).
  \begin{alignat*}{2}
    (x_1y_1)^2 & = rsrs &&{}= 1\\
    & = (rsrs)s &&{}= s\\
    & = (rsr)r^{-1} &&{}= sr^{-1}\\
    & = rs &&{}= sr^{-1}\\
    & = x_1y_1 &&{}= y_1x_1^{-1}
  \end{alignat*}
\item
  Let \(X_{2n}\) be the group whose presentation is
  \[
  X_{2n} = \langle x,y\mid x^n = y^2 = 1, xy = yx^2\rangle.
  \]
  \begin{enumerate}[label = (\alph*)]
  \item
    Show that if \(n = 3k\), then \(X_{2n}\) has order \(6\), and it has the
    same generators and relations as \(D_6\) when \(x\) is replaced by \(r\)
    and \(y\) by \(s\).
    \par\smallskip
    We can wrtie \(x\) as
    \[
    xy^2 = xyy = yx^2y = yyx^4\Rightarrow x = x^4\Rightarrow x^3 = 1
    \]
    Therefore, the order of \(x\) is three.
    Now \(x^n = r^{3k} = (r^3)^k = 1^k = 1\) and \(y^2 = s^2 = 1\).
    The distinct elements of \(X_{2n} = \{1,x,x^2,y,yx,yx^2\}\) so the
    \(\lvert X_{2n}\rvert = 6\).
    When \(x\) and \(y\) are replaced by \(r\) and \(s\), we obtain the same
    generators as \(D_6\).
  \item
    Show that if \((3,n) = 1\), then \(x\) satisfies the additional relation:
    \(x = 1\).
    In this case, deduce that \(X_{2n}\) has order \(2\).
    [Use the fact that \(x^n = 1\) and \(x^3 = 1\).]
    \par\smallskip
    
  \end{enumerate}
\item
  Let \(Y\) be the group whose presentation is
  \[
  Y = \langle u,v\mid u^4 = v^3 = 1, uv = v^2u^2\rangle.
  \]
  \begin{enumerate}[label = (\alph*), ref = \theenumi (\alph*)]
  \item
    \label{18a}
    Show that \(v^2 = v^{-1}\).
    [Use the relation \(v^3 = 1\).]
  \item
    \label{18b}
    Show that \(v\) commutes with \(u^3\).
    [Show that \(v^2u^3v = u^3\) by writing the left hand side as
    \((v^2u^2)(uv)\) and using the relations to reduce this to the right hand
    side.
    Then use \cref{18a}.]
  \item
    \label{18c}
    Show that \(v\) commutes with \(u\).
    [Show that \(u^9 = u\) and then use \cref{18b}.]
  \item
    \label{18d}
    Show that \(uv = 1\).
    [Use \cref{18c} and the last relation.]
  \item
    Show that \(u = 1\), deduce that \(v = 1\), and conclude that \(Y = 1\).
    [Use \cref{18d} and the equation \(u^4v^3 = 1\).]
  \end{enumerate}
\end{enumerate}
%%% Local Variables:
%%% mode: latex
%%% TeX-master: t
%%% End:
